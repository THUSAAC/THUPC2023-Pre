\documentclass{beamer}

\usepackage[UTF8,noindent]{ctexcap}
\usepackage{color}%引入颜色
\usetheme{Berlin}
\usecolortheme{spruce}
\usepackage{graphicx}%引入插图
\usepackage{ulem}%删除线
\usepackage{tikz}
\usepackage{amsmath}
\usefonttheme[onlymath]{serif}

\setbeamertemplate{theorems}[numbered]

\title{THUPC2022 背包}
\author{Itst}
\institute{THU\ IIIS}
\begin{document}
\begin{frame}
\titlepage
\end{frame}
\section{简要题意}
\begin{frame}{简要题意}
	有 $n$ 种物品,第 $i$ 种物品单个体积为 $v_i$、价值为 $c_i$。

	$q$ 次询问,每次给出背包的容积 $V$,你需要选择若干个物品,每种物品可以选择任意多个(也可以不选),在选出物品的体积的和恰好为 $V$ 的前提下最大化选出物品的价值的和。

	$1 \le n \le 50$,$1 \le v_i \le 10^5$,$1 \le c_i \le 10^6$,$1 \le q \le 10^5$,$10^{11} \le V \le 10^{12}$。
\end{frame}
\section{解法}
\begin{frame}{解法}
	$V$ 很大,一个很直观的想法是:大部分选 $\frac{c_i}{v_i}$ 最大的物品,边角用其他物品填满。我们先假设 $V$ 非常非常非常大并考察一个解法,再证明题目中 $V$ 的下界保证了该算法的正确性。\pause

	设 $(c_0,v_0)$ 为 $\frac{c_i}{v_i}$ 最大的物品的参数。根据 $\bmod\ v_0$ 的不同,边角有不同方案,而 $V$ 足够大时,$\bmod\ v_0$ 相同的询问的边角选择方案总会相同。因此计算出每种边角的最优方案即可 $O(1)$ 得到答案。
\end{frame}
\begin{frame}{解法}
	为此,设 $f_i$ 为最优物品以外的物品的完全背包数组,并设 $g_i = \max_{k \ge 0} f_{i + kv_0} - kc_0$。那么 $\bmod b_0 = i$ 的边角最优方案的收益就是 $g_i$。注意每当 $k$ 增大 $1$ 时会占用一个最优物品所以需要减掉一个 $c_0$。\pause

	沿用完全背包的 dp 思路计算 $g_i$。加入一个 $(c_i,v_i)$,转移为 $$g_{(j + v_i) \bmod v_0} \leftarrow \max\left(g_{(j + v_i) \bmod v_0} , g_j + c_i - \lfloor \frac{j + v_i}{v_0} \rfloor c_0\right).$$\pause
	
	注意到转移成环,但由于 $(c_0,v_0)$ 的最优性,转移不可能无限进行,所以 dp 规则是良定义的。转移构成若干个圈,沿着每个圈转移两次即可计算出正确的 dp 值。

	以上算法复杂度 $O(q + n \max v_i)$。
\end{frame}
\begin{frame}{正确性证明}
	我们还需要证明 $V$ 确实足够大,也就是说 $g_i$ 对应的方案的容积不超过 $V$ 的下界。

	注意到上面的动态规划也可以视为最短路,根据最短路不重复经过单点的性质,每个 $g_i$ 对应的方案都一定有不超过 $v_0$ 个物品,所以总容积不超过 $\max v_i^2 = 10^{10}$,因此在本题设定下该算法是正确的。
\end{frame}
\end{document}