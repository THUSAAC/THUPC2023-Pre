\documentclass{beamer}

\usepackage[UTF8,noindent]{ctexcap}
\usepackage{color}%引入颜色
\usetheme{Berlin}
\usecolortheme{spruce}
\usepackage{graphicx}%引入插图
\usepackage{ulem}%删除线
\usepackage{tikz}
\usepackage{amsmath}
\usefonttheme[onlymath]{serif}

\setbeamertemplate{theorems}[numbered]

\title{THUPC2022 快速最小公倍数变换}
\author{Itst}
\institute{THU\ IIIS}
\begin{document}
\begin{frame}
\titlepage
\end{frame}
\section{简要题意}
\begin{frame}{简要题意}
	给定 $n$ 个数 $a_1,a_2,\cdots,a_n$,选择两个数删掉并将它们的和加入,求所有的方案的最小公倍数的和,对 $998244353$ 取模。

	$n \le 5 \times 10^5, 1 \le a_i \le 10^6$
\end{frame}
\section{解法}
\begin{frame}{定义}
	对于正整数 $A$ 和素数 $p$,定义 $v_p(A)$ 为 $A$ 的唯一质因子分解中 $p$ 的出现次数。因此 $p^{v_p(A)} \mid A$ 而 $p^{v_p(A)+1} \nmid A$。

	定义 $M(p)$ 和 $m(p)$ 为 $\{v_p(a_1),\cdots,v_p(a_n)\}$ 中的最大值和非严格次大值,即当最大值出现超过一次时,$m(p) = M(p)$。\pause

	那么 $\prod_{p \in \texttt{Prime}} p^{M(p)}$ 就是整个序列的最小公倍数。接下来删掉两个数并加上一个数时,可以仅考虑 $M(p)$ 的变化量。
\end{frame}
\begin{frame}{观察}
	删除 $a_i$ 和 $a_j$ 并加入 $a_i+a_j$ 时,我们期待 $M(p)$ 的变化可以由 \begin{itemize}
		\item 仅和 $a_i$ 相关的量;
		\item 仅和 $a_j$ 相关的量;
		\item 仅和 $a_i+a_j$ 相关的量
	\end{itemize}
	这三部分刻画,这样贡献的独立性更方便我们进行卷积操作。
\end{frame}
\begin{frame}{定理}
	\begin{theorem}
		对于任意 $a_i$ 和 $a_j$,设将 $a_i$ 和 $a_j$ 删去并加入 $a_i+a_j$ 后整个序列的最小公倍数中 $p$ 的指数的变化量为 $\Delta M_p(a_i,a_j)$,则 \begin{align*}
			\Delta M_p(a_i,a_j) = & [M(p) = v_p(a_i)](m(p) - M(p)) \\
			 + & [M(p) = v_p(a_j)](m(p) - M(p)) \\ 
			 + & \max(v_p(a_i+a_j) - M(p) , 0).
		\end{align*}
	\end{theorem}\pause
	也就是说,$\Delta M_p(a_i,a_j)$ 直接由删去 $a_i$ 造成的指数减小、删去 $a_j$ 造成的指数减小和加入 $a_i+a_j$ 造成的指数增大三部分贡献。这三部分贡献的基准都是 $M(p)$,这是非常反直觉的,而在删除超过两个数时类似结论不成立。
\end{frame}
\begin{frame}{证明}
	\begin{align*}
		\Delta M_p(a_i,a_j) = & [M(p) = v_p(a_i)](m(p) - M(p)) \\
		 + & [M(p) = v_p(a_j)](m(p) - M(p)) \\ 
		 + & \max(v_p(a_i+a_j) - M(p) , 0).
	\end{align*}
	
	~\\
	
	定义 $f_i = [M(p) = v_p(a_i)] , f_j = [M(p) = v_p(a_j)] , f_k = [v_p(a_i+a_j) \ge M(p)] , f_l = [m(p) = M(p)]$。分类讨论:

	~\\
	
	\begin{itemize}
		\item $f_i=f_j= 0$:上式显然成立。
	\end{itemize}
\end{frame}
\begin{frame}{证明}
	\begin{align*}
		\Delta M_p(a_i,a_j) = & [M(p) = v_p(a_i)](m(p) - M(p)) \\
		 + & [M(p) = v_p(a_j)](m(p) - M(p)) \\ 
		 + & \max(v_p(a_i+a_j) - M(p) , 0).
	\end{align*}
	
	~\\
	
	定义 $f_i = [M(p) = v_p(a_i)] , f_j = [M(p) = v_p(a_j)] , f_k = [v_p(a_i+a_j) \ge M(p)]$。分类讨论:
	
	~\\

	\begin{itemize}
		\item $f_i = 1, f_j = 0$:此时 $v_p(a_i) = M(p) > v_p(a_j)$,故 $v_p(a_i+a_j) = v_p(a_j)$,$\{v_p(a_i)\}$ 集合只删掉了一个最大值。因此 $f_k = 0$ 且 $\Delta M_p(a_i, a_j) = (m(p) - M(p))$,与上式相符。
	\end{itemize}
\end{frame}
\begin{frame}{证明}
	\begin{align*}
		\Delta M_p(a_i,a_j) = & [M(p) = v_p(a_i)](m(p) - M(p)) \\
		 + & [M(p) = v_p(a_j)](m(p) - M(p)) \\ 
		 + & \max(v_p(a_i+a_j) - M(p) , 0).
	\end{align*}
	
	~\\
	
	定义 $f_i = [M(p) = v_p(a_i)] , f_j = [M(p) = v_p(a_j)] , f_k = [v_p(a_i+a_j) \ge M(p)] , f_l = [m(p) = M(p)]$。分类讨论:
	
	~\\

	\begin{itemize}
		\item $f_i = f_j = 1$:此时 $m(p) = M(p)$,且 $v_p(a_i+a_j) \ge M(p)$。因此 $f_k = 1$ 且 $\Delta M_p(a_i, a_j) = v_p(a_i+a_j) - M(p)$,与上式相符。
	\end{itemize}
\end{frame}
\begin{frame}{解法}
	由此,设 $$f(i) = \prod_{p \in \texttt{Prime}} p^{[M(p) = v_p(i)](m(p) - M(p))},$$ $$g(j) = \prod_{p \in \texttt{Prime}} p^{\max(v_p(j) - M(p) , 0)},$$则答案为 $$\sum_{1 \le i < j \le n} f(a_i)f(a_j)g(a_i+a_j).$$
\end{frame}
\begin{frame}{解法}
	此时对值域维度卷积的方法较为显然。定义 
	$$F(x) = \sum_{i=1}^n f(a_i) x^{a_i},$$ 则有 $$[x^p]F^2(x) = \sum_{1 \le i,j \le n} f(a_i)f(a_j)[a_i+a_j = p].$$
	计算 $\frac{1}{2} \left(F^2(x) \cdot \sum_{i \ge 0} g(i) x^i \right)$ 并去除 $i=j$ 的多余贡献即可。

	使用 $O(\max a_i) - O(\log \max a_i)$ 分解质因数方法,复杂度 $O((n + \max a_i) \log (\max a_i))$。
\end{frame}
\end{document}