\documentclass{beamer}

\usepackage[UTF8,noindent]{ctexcap}
\usepackage{color}%引入颜色
\usetheme{Berlin}
\usecolortheme{spruce}
\usepackage{graphicx}%引入插图
\usepackage{ulem}%删除线
\usepackage{tikz}
\usepackage{amsmath}
\usefonttheme[onlymath]{serif}

\setbeamertemplate{theorems}[numbered]

\title{THUPC2022 速战速决}
\author{Itst}
\institute{THU\ IIIS}
\begin{document}
\begin{frame}
\titlepage
\end{frame}
\section{简要题意}
\begin{frame}{简要题意}
	$n$ 种牌,每种牌各两张,双方初始获得 $n$ 张手牌。双方轮流在栈里面放牌,当栈里有两张相同的牌时,这两张牌及其中间的牌从栈中移动到放牌的人的手牌里。谁没手牌谁就输了。

	后手的策略为一个队列,每次一定出队头,获得的手牌放在队尾,初始手牌的出牌顺序给定。先手可以任意出牌,求先手的最小操作次数策略使得先手胜利。

	$n \le 3 \times 10^5$。

	~\\

	Bonus:先手获得的手牌数小于后手?
\end{frame}
\section{解法}
\begin{frame}{分析}
	操作次数的下界显然是 $n$,而这要求后手吃不到牌。因此我们先考虑怎样才能让后手永远吃不到牌。\pause

	重要的观察是,当栈底那张牌的另一张在后手手上,那么 TA 肯定会吃到这张牌,因为你的策略无法撼动这张牌在栈底的地位。\pause

	类似地,当栈底那张牌的另一张在先手手上,那么场面对先手来说是非常有利的,因为在任意先手的操作时刻,先手都可以选择打出这一张牌将场面清空。为了有效控场,这张王牌必然是越晚用越好。
\end{frame}
\begin{frame}{分析}
	这样,在初始手牌有对子时,容易得到一个策略:初始放入手上的一个对子中的一张,而每当先手发现后手可以吃到牌的时候就把栈清空,否则放入任意的其他牌。\pause

	此时由两个事实可以说明这个策略总是可以让后手吃不到牌:\begin{itemize}
		\item 先手清空后,后手放入的落在栈底的牌在原来的栈中,所以也就在先手的手牌里。所以先手可以一直保持对整个栈的控制;
		\item 先手不想清空时,由于双方初始手牌数相等且后手永远吃不到牌,先手总是能找到不是栈底的牌打出去。
	\end{itemize}
\end{frame}
\begin{frame}{分析}
	以上策略要求初始手牌中至少有一个对子,没有考虑先手恰好有 $n$ 种不同的牌的情况。
	
	这个情况下,栈底的牌总是会被后手吃,因此最小操作数至少是 $n+2$,而构造到 $n+2$ 是简单的:设后手队列为 $1,2,3,\cdots,n$,则按照 $n,1,2,\cdots,n-1$ 顺序打出手牌,最后对手手上就只有一对 $n$。
\end{frame}
\end{document}