\documentclass{beamer}

\usepackage[UTF8,noindent]{ctexcap}
\usepackage{color}%引入颜色
\usetheme{Berlin}
\usecolortheme{spruce}
\usepackage{graphicx}%引入插图
\usepackage{ulem}%删除线
\usepackage{tikz}
\usepackage{amsmath}
\usefonttheme[onlymath]{serif}

\setbeamertemplate{theorems}[numbered]

\title{THUPC2022 众数}
\author{Itst}
\institute{THU\ IIIS}
\begin{document}
\begin{frame}
\titlepage
\end{frame}
\section{简要题意}
\begin{frame}{简要题意}
	找到一个序列,其中有 $a_i$ 个 $i(1 \le i \le n)$,且前缀众数和尽可能大。

	$1 \le a_i,n \le 10^5$
\end{frame}
\section{解法}
\begin{frame}{解法}
	直观的想法是,让 $n$ 在尽可能多的前缀中是众数,其次让 $n-1$ 在尽可能多的前缀中是众数,以此类推。

	那么一个很显然的贪心是,先放 $a_n$ 个 $n$,然后把其他数放到 $\min(a_n,a_i)$ 个;然后再放 $a_{n-1}$,再把其他数放到 $\min(\max(a_n,a_{n-1}),a_i)$ 个,以此类推。\pause

	如上直接做是平方的。上面的贪心策略实际上跟这样的贪心策略是等价的:先从大到小将每个数的第一次出现放入序列,然后将它们的第二次出现放入序列,以此类推。

	故可以枚举这个第几次出现,然后用最大值乘元素个数贡献答案。用 set 维护,复杂度 $\tilde{O}(\max a_i + n)$。\pause 

	如上贪心算法的正确性证明,可以考虑众数 $\ge i$ 的前缀个数最多为多少,并发现如上构造方法达到了该上界。
\end{frame}
\end{document}